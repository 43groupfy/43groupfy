\documentclass[12 pt]{article}

\usepackage[utf8]{inputenc}
\usepackage[english]{babel}
\usepackage{ragged2e}
\usepackage{mathtools, amssymb, amsthm}

\usepackage{pgfplots}
\pgfplotsset{compat=1.18}
\usetikzlibrary{shapes, positioning, arrows.meta}

\usepackage{geometry}
\geometry{margin=1in}

\usepackage{hyperref}
\hypersetup{
    colorlinks=true,
    linkcolor=blue,
    filecolor=magenta,
    urlcolor=cyan,
    pdftitle={Classical Mechanics Studied Using Trees},
    }
\urlstyle{same}

\setlength{\parskip}{1em}

\title{Classical Mechanics Studied Using Trees}
\author{Booodaness}
\date{\today}

\clearpage

\begin{document}

\pagenumbering{roman}

\begin{titlepage}
\maketitle
\begin{abstract}
In this document, we examine the logical structure of classical physics using graph theory. Namely, we model the trajectories of discrete systems in state space as trees with update structure and some underlying axioms motivated by classical mechanics.
\end{abstract}
\end{titlepage}

\setcounter{page}{2}

\tableofcontents

\clearpage
\pagenumbering{arabic}

\section{Graph Theory: Motivation}

\subsection{Discrete systems}

Suppose there is a system $\mathcal{S}$ which updates discretely with time. Let every step i.e. discrete instant in time be labelled by an integer $n$ and the state of $\mathcal{S}$ at the concerned instant as $\sigma \left( n \right)$.

We can imagine that as time passes, $n$ increments in `jumps', as $n, n+1, n+2 \dots$. In accordance, $\sigma \left( n \right)$ updates as $\sigma \left( n \right), \sigma \left( n+1 \right), \sigma \left( n+2 \right), \dots$.

The coordinates of $\mathcal{S}$ at any instant $n$ are $\left( n, \sigma \left( n \right) \right)$. Let the set that $n$ belongs to be called its \emph{time domain}, $T \subseteq \mathbb{Z}$. Also, let $\sigma : T \mapsto \mathbb{R}$. Then, every coordinate $\left( n, \sigma \left( n \right) \right)$ belongs to $\Sigma = T \times \mathbb{R}$, called the \emph{state space} of $\mathcal{S}$.

The set of all possible coordinates of $\mathcal{S}$ (at different instants) in its state space $\Sigma$ is called its \emph{trajectory}, or \emph{history}. In set theoretic notation, the trajectory of $\mathcal{S}$ is $\left\{ \left( n, \sigma \left( n \right) \right) \vert n \in T \right\}$.

A visual way of representing the trajectory of a system is to plot the state $\sigma \left( n \right)$ as $n$ updates. For example, if a system's state starts at $0$ and updates by $1$ every unit time interval, its trajectory looks as shown:

\begin{figure}[h]
\label{fig:1}
\centering
\begin{tikzpicture}
\begin{axis}[
    axis lines = center,
    xlabel style={below right},
    ylabel style={above left},
    xlabel = $n$,
    ylabel = {$\sigma \left( n \right) = n$},
    xmin=0,
    xmax=3.5,
    ymin=0,
    ymax=3.5,
    xtick={1,2,3},
    ytick={1,2,3},
    xmajorgrids = true
]

\addplot[
    color = blue,
    mark = *,
    mark options = {solid}
    ]
    coordinates {
    (0,0)(1,1)(2,2)(3,3)
    };

\end{axis}
\end{tikzpicture}
\caption{Evolution of $\mathcal{S}$}
\end{figure}

In the above figure, the vertical lines indicate that $n$, shown along the x-axis, is discrete.

Such a diagram may be called a \emph{state update plot}.

\clearpage

\subsection{Update structure}

Much of a system's evolution can be pictured using its state update plot. However, it does not contain all the information necessary to understand the system's evolution. Let us see why.

What figure $\ref{fig:1}$ is really saying is that \emph{as $n$ updates discretely}, $\sigma$ updates discretely according to the diagram. While discerning the nature of $\mathcal{S}$ from its state update plot, we consider the update of $n$ with time implicit.

But the mathematical language of physics should make things as explicit as possible, for a deeper description and minimum ambiguity. Now, how do we make the updating of $n$ explicit in a diagram?

This is where graph theory comes in. As discussed in the previous section, we imagine the motion of $\mathcal{S}$ in $\Sigma$ as discrete updating of the coordinates in the manner,

$\dots, \left( n, \sigma \left( n \right) \right), \left( n+1, \sigma \left( n+1 \right) \right), \dots$

Firstly, let us call the coordinates at a given instant the \emph{node} at that instant, $\pmb{q} \left( n \right) = \left( n, \sigma \left( n \right) \right), \pmb{q} \left( n \right) \in \Sigma$.

In the evolution of $\mathcal{S}$, every node has a unique node immediately after it. To make this idea concrete, we define a function $E : \Sigma \mapsto \Sigma$,

$E = \left\{ \pmb{q} \left( n \right), \pmb{q} \left( n+1 \right) \vert n \in T \right\}$

Since $E$ is a function, every node $\pmb{q} \left( n \right)$ has a unique node it updates to, $E \left( \pmb{q} \left( n \right) \right) = \pmb{q} \left( n+1 \right)$.

To represent this information in the state update plot, we think of the position of $\mathcal{S}$ at each instant as a node, and every ordered pair of nodes as an arrow connecting the nodes. In other words, the arrows \emph{explicitly} represent how each state is mapped to its next state as $n$ updates.

\clearpage

\begin{figure}[h]
\label{fig:2}
\centering
\begin{tikzpicture}
\tikzstyle{node} = [draw, circle, fill=white]
\tikzstyle{arrow} = [draw, -Latex, color=blue]
\begin{axis}[
    axis lines = center,
    xlabel style={below right},
    ylabel style={above left},
    xlabel = $n$,
    ylabel = {$\sigma \left( n \right) = n$},
    xmin=-0.5,
    xmax=3.5,
    ymin=-0.5,
    ymax=3.5,
    xtick={1,2,3},
    ytick={1,2,3},
    xmajorgrids = true
]

\node[node] (0) at (axis cs:0,0) {$\pmb{q}_0$};
\node[node] (1) at (axis cs:1,1) {$\pmb{q}_1$};
\node[node] (2) at (axis cs:2,2) {$\pmb{q}_2$};
\node[node] (3) at (axis cs:3,3) {$\pmb{q}_3$};

\draw[arrow] (0) -- (1);
\draw[arrow] (1) -- (2);
\draw[arrow] (2) -- (3);

\end{axis}
\end{tikzpicture}
\caption{Update structure in the evolution of $\mathcal{S}$}
\end{figure}

(Here, subscripts on $\pmb{q}$ represent the corresponding values of $n$).

In the above picture of a system's evolution, there are two kinds of information:

\begin{enumerate}
  \item \emph{Nodes} or \emph{vertices}, representing coordinates $\pmb{q} \left( n \right)$

  \item \emph{Arrows} or \emph{edges}, representing state updates with their direction
\end{enumerate}

The set of all nodes may be written as $V = \left\{ \pmb{q} \left( n \right) \vert n \in T \right\}$, and the set of edges as $E$, which we have already defined as a function from $\Sigma$ to itself. Then, all the information contained in figure $\ref{fig:2}$ may be packed into a single structure, called a \emph{graph},

$$G = \left( V, E \right)$$

Notice that $V \subseteq \Sigma$. Therefore, $E : \Sigma \mapsto \Sigma \implies E : V \mapsto V$. This means that a graph is simply a set of vertices equipped with a function on itself. However, in common terminology, there is a condition for such a structure to be a graph: it cannot contain any ordered pairs of some vertex and itself. I.e., no arrow in a graph points from one vertex to itself.

\clearpage

\section{Classical Mechanics and Trees}

\subsection{Some principles of classical physics}

\end{document}
